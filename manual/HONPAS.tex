% Manual for the HONPAS-SIESTA program
%
% To generate the printed version:
%
% latex HONPAS
% makeindex   HONPAS  (Optional if you have a current siesta.ind)
% latex HONPAS
% [ dvips  HONPAS]
%
%
\documentclass[11pt]{article}
\usepackage{makeidx}
\usepackage{url}
\usepackage{hyperref}
\usepackage{hyperref}
% Hyperref setup
\hypersetup{
    hyperindex=true,
    bookmarksopen=true,
    bookmarksopenlevel=2,
}

\tolerance 10000
\textheight 22cm
\textwidth 16cm
\oddsidemargin 1mm
\topmargin -15mm

\makeindex

\parindent=0cm
\baselineskip=14pt
\parskip 5pt

\begin{document}

% TITLE PAGE --------------------------------------------------------------

\begin{titlepage}

\begin{center}

\vspace{3cm}
{\Huge {\bf Order-N quantum chemistry Package for large scale \it ab initio \bf Simulation  }}

\vspace{3cm}
{\Huge {\bf HONPAS }}

\vspace{1cm}
{\Large Version 1 }   \\
\vspace{1cm}
{\Large November, 2013 }

\vspace{1.5cm}

\hbox{ \hskip 1.5cm
\begin{tabular}{ll}

{\Large Jinlong Yang} &
   \Large{\it University of Science and Technology of China} \\ \\

{\Large Hongjun Xiang} &
   \Large{\it Fudan University} \\ \\

{\Large Honghui Shang} &
   \Large{\it Fritz Haber Institute} \\ \\
   
{\Large Xinming Qin} &
   \Large{\it University of Science and Technology of China} \\ \\   
\end{tabular}
}




\end{center}

\end{titlepage}

% END TITLE PAGE --------------------------------------------------------------

\newpage

\tableofcontents

\newpage

\section{INTRODUCTION}



{\sc HONPAS} is based on {\sc SIESTA}, but adding:

\begin{itemize}
\item Density matrix purification methods to make construction of the density matrix in $O(N)$ operations

\item Maximally localized Wannier functions

\item Density-matrix perturbation theory

\item Hartree-Fock method for molecule or solid 

\item Hybrid functionals: B3LYP, PBE0, HSE06  for molecule or solid 

\item MP2 method for molecule or solid 

\item All the new feathers are compiled for serial or parallel execution (under MPI).

\end{itemize}


\vspace{0.5cm}
{\large {\bf References:} }

\begin{itemize}

\item  "Spin-unrestricted linear-scaling electronic structure theory and its application to magnetic carbon-doped boron nitride nanotubes"

Xiang, HJ, Liang, WZ, Yang, JL, Hou, JG and Zhu, QS. J. Chem. Phys., {\bf 123}, 124105  (2005)

{\bf Density matrix purification methods implemented in {\sc HONPAS} }


\item "Linear scaling calculation of maximally localized Wannier functions with atomic basis set"

Xiang, HJ, Yang, JL, Hou, JG and Zhu, QS. J. Chem. Phys., {\bf 126}, 244707 (2007)

{\bf Calculation of maximally localized Wannier functions in 
{\sc HONPAS} }

\item "Linear-scaling density matrix perturbation treatment of electric fields in solids"

Xiang, HJ, Yang, JL, Hou, JG and Zhu, QS.  Phys. Rev. Lett., {\bf 97}, 266402 (2006)

{\bf Density matrix perturbation theory implemented in {\sc HONPAS} } 




%shanghui
\item "Implementation of Exact Exchange with Numerical Atomic Orbitals"
H. H. Shang, Z. Y. Li, and J. L. Yang, J. Phys. Chem. A 114, 1039 (2010)

{\bf HF and B3LYP in {\sc HONPAS}}


\item "Implementation of screened hybrid density functional for periodic systems with numerical atomic orbitals: Basis function fitting and integral screening"
H. H. Shang, Z. Y. Li, and J. L. Yang, J. Chem. Phys. {\bf 135}, 034110 (2011)

{\bf HSE06 in {\sc HONPAS}}


\item "Linear scaling electronic structure calculations with numerical atomic basis set"

H. H. Shang, Z. Y. Li, and J. L. Yang, Int. Rev. Phys. Chem. {\bf 29}, 665 (2010).

{\bf Review for O(N) method with numerical atomic basis set}

\end{itemize}


\section{COMPILATION}

If you just want to compile the program, go to {\tt Obj} and issue the command:

\begin{verbatim}
  sh ../Src/obj_setup.sh
\end{verbatim}

Then use:

\begin{verbatim}
  make HONPAS
\end{verbatim}

The executable should work for any job. 



\section{DETAILED DESCRIPTION OF PROGRAM OPTIONS}

As we based on {\sc Siesta}, the input files are the same except the following: 

\subsection{Order(N) calculations}
\begin{description}
\itemsep 10pt
\parsep 0pt
\item[{\bf ON purification }] 
Choice of purification method
\begin{itemize}
\item {\tt TC2}
\end{itemize}
\end{description}

\subsection{Maximally localized Wannier functions}
\begin{description}
\itemsep 10pt
\parsep 0pt
\item[{\bf MLWF }]
Generate maximally localized Wannier functions for molecules.
\end{description}



\vspace{5pt}
\subsection{Hartree-Fock for exact exchange energy}

\begin{description}
\itemsep 10pt
\parsep 0pt

\item[{\bf HF}] ({\it string}): 
Make a Hartree-Fock calculation for molecule or solid

\begin{itemize} 
\item {\tt numeric}  
Using isf-poisson solver to calculate ERI numerically. 

\item {\tt NAO2GTO}
Using more efficient NAO2GTO scheme to calculate ERI.
\end{itemize}
{\it Default value:} {\tt NAO2GTO}

\end{description}




\vspace{5pt}
\subsection{Hybrid Exchange-correlation functionals}

\begin{description}
\itemsep 10pt
\parsep 0pt

\item[{\bf XC.functional}] ({\it string}):
\index{XC.functional@{\bf XC.functional}}
Exchange-correlation functional type.  As we add hybrid
functionals, so the default is 
{\tt GGA} (Generalized Gradient Approximation).


{\it Default value:} {\tt GGA}


\item[{\bf XC.authors}] ({\it string}):
\index{XC.authors@{\bf XC.authors}}
Particular parametrization of the
exchange-correlation functional. Options are:
\begin{itemize}


\item {\tt B3LYP}   \index{B3LYP}

Ref: Becke, A. D. J. Chem. Phys. {\bf 98}, 5648 (1993)


\item {\tt HSE06}   \index{HSE06}

Ref: A. V. Krukau, O. A. Vydrov, A. F. Izmaylov and G. E. Scuseria,
J. Chem. Phys. {\bf 125}, 224106 (2006).


\end{itemize}

{\it Default value:} {\tt HSE06}



\end{description}


\vspace{5pt}
\subsection{Second-order M\o{}ller–Plesset perturbation for
correlation energy}
\begin{description}
\itemsep 10pt
\parsep 0pt

\item[{\bf MP2}] 
Make a Hartree-Fock calculation for molecule or solid.

{\it Default value:} {\tt MP2}

\end{description}


\







\addcontentsline{toc}{section}{Index}
\printindex

\end{document}

