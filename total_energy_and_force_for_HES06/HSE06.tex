\documentclass[aps,preprint,12pt]{revtex4-1}
%\documentclass[aip,preprint,12pt]{revtex4-1}
\usepackage{multirow}
\usepackage{amssymb}
\usepackage{amsmath}
\usepackage{graphicx}
\usepackage{algorithmic}
\usepackage{algorithm}

\begin{document}

\title{HSE06 functional in siesta}
\author{ Honghui Shang}

\affiliation{FHI-Berlin}

\date{\today}

\begin{abstract}
The total energy and force equations for HSE06 functional in siesta is written here. By this way, further extension (mp2, hessian) can be done
based on this. 
\end{abstract}
\maketitle


\section{METHODS}
\subsection{HSE total energy}
In a general fore, no matter what basis set you use or what interaction you use to 
describe the electron-nuclue interaction, we have 




\subsection{HSE force}
For un-spin-polarized systems (nspin=1), the Gradient is divided into two terms:
\begin{equation}
 \dfrac{\partial{E_{HFX}}}{\partial{R_{I}}}
=-\frac{1}{2}\sum_{\mu\lambda}\sum_{\mathbf{G}}\dfrac{P_{\mu\lambda}^{\mathbf{G}}}{\partial{R_{I}}}\sum_{\nu\sigma}\sum_{\mathbf{N,H}}P_{\nu\sigma}^\mathbf{H-N}\mathbf{[(\chi_{\mu}^{0}\chi_{\nu}^{N}|\chi_{\lambda}^{G}\chi_{\sigma}^{H})]}
\end{equation}
\[
 -\frac{1}{4}\sum_{\mu\lambda}\sum_{\mathbf{G}}P_{\mu\lambda}^{\mathbf{G}}\sum_{\nu\sigma}\sum_{\mathbf{N,H}}P_{\nu\sigma}^\mathbf{H-N}\mathbf{ \dfrac{\partial{(\chi_{\mu}^{0}\chi_{\nu}^{N}|\chi_{\lambda}^{G}\chi_{\sigma}^{H})}}{\partial{R_{I}}} }
\]


The first term can be calculated in the orthogonalization force:
\begin{equation}
 \sum_{\mu\nu}{F_{\mu\nu}\dfrac{\partial{P_{\mu\nu}}}{\partial{R_{I}}}}=
-\sum_{\mu\nu}{E_{\mu\nu}\dfrac{\partial{S_{\mu\nu}}}{\partial{R_{I}}}}
\end{equation}
where
\[
 E_{\mu\nu}=\sum_{i}{c_{\mu i}c_{\nu i}n_i\varepsilon{i} }
\]


The second term need the gradient of ERIs. In the following, we will deal with this term: 
\[
 F_{\mathbf{R_I}}=\frac{1}{4}\sum_{\mu\lambda}\sum_{\mathbf{G}}P_{\mu\lambda}^{\mathbf{G}}\sum_{\nu\sigma}\sum_{\mathbf{N,H}}P_{\nu\sigma}^\mathbf{H-N}\mathbf{ \dfrac{\partial{(\chi_{\mu}^{0}\chi_{\nu}^{N}|\chi_{\lambda}^{G}\chi_{\sigma}^{H})}}{\partial{R_{I}}} }
\]
\[
 =\frac{1}{4}\sum_{\mu\lambda}\sum_{\mathbf{G}}P_{\mu\lambda}^{\mathbf{G}}\sum_{\nu\sigma}\sum_{\mathbf{N,H}}P_{\nu\sigma}^\mathbf{H-N}
\]
\begin{equation}
\times\mathbf{[ (\dfrac{\chi_{\mu}^{0}}{\partial{R_{I}}}\chi_{\nu}^{N}|\chi_{\lambda}^{G}\chi_{\sigma}^{H})
+(\chi_{\mu}^{0}\dfrac{\chi_{\nu}^{N}}{\partial{R_{I}}}|\chi_{\lambda}^{G}\chi_{\sigma}^{H}) 
+(\chi_{\mu}^{0}\chi_{\nu}^{N}|\dfrac{\chi_{\lambda}^{G}}{\partial{R_{I}}}\chi_{\sigma}^{H}) 
+(\chi_{\mu}^{0}\chi_{\nu}^{N}|\chi_{\lambda}^{G}\dfrac{\chi_{\sigma}^{H}}{\partial{R_{I}}})   ]}
\end{equation}




\section{CONCLUSIONS}

\begin{acknowledgments}

\end{acknowledgments}


\begin{thebibliography} {99}

\bibitem{Parr} R. G. Parr and W. Yang, \emph{Density Functional Theory of Atoms and Molecules }(Oxford University Press, New York,1989)

\bibitem{DFT_1} P. Hohenberg and W. Kohn, Phys. Rev. B {\bf 136}, 864 (1964).
\bibitem{DFT_2} W. Kohn and L. Sham, Phys. Rev. {\bf 140}, 1133 (1965).


\end{thebibliography}



\end{document}
